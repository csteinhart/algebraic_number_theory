\begin{Bsp}
Same situation as in Ex. 2.5. $\B_1=\{1, \sqrt{D}\} \subseteq B$. Consider:
\begin{align*}
&\alpha = \frac{1}{2}(1+ \sqrt{D}) \Rightarrow 2 \alpha = 1+ \sqrt{D}\\
\Rightarrow & (2\alpha-1)^2=D \Rightarrow 4 \alpha^2-4\alpha+1=D\\
\Rightarrow & f_\alpha(X)=X^2-X+\frac{1-D}{4}
\end{align*}
Hence if $D \equiv 1 \mod 4 \Rightarrow \alpha \in B$ and $\B_1$ is not an integral basis.
\end{Bsp}

\begin{Prop}
Let $D \in \Z$, $D$ square-free, $D \not = 0,1, B:= $ integral closure of $\Z$ in $\Q(\sqrt{D})=L$.
\begin{enumerate}[i)]
\item $D \equiv 2,3 \mod 4 \Rightarrow \{1, \sqrt{D}\}$ is an integral basis of $B / \Z$ in particular $B= \Z[\sqrt{D}]$.
\item $D \equiv 1 \mod 4 \Rightarrow \{1, \frac{1}{2}(\sqrt{D}+1)\}$ is an integral basis of $B / \Z$. and $B = \Z[\frac{1}{2} (1+\sqrt{D})]$.
\end{enumerate}
\end{Prop}

\begin{Bew}
Consider $\alpha = a+b \sqrt{D} \in \Q(\sqrt{D})$ with $a,b, \in \Q$.\\
$\Rightarrow f_\alpha = X^2-2aX+a^2-b^2D.$\\
Rem 2.1: $\alpha \in B \iff f_\alpha \in \Z[X] \iff 2a \in \Z$ and $a^2-b^2D \in \Z$.
\begin{enumerate}[(1)]
\item \underline{Show}: $\alpha \in B \Rightarrow 2b \in \Z$.\\
$\alpha \in B \Rightarrow 4a^2-4b^2D = 4z$ with $z \in \Z$. Write $b= \frac{p}{q}$ with $p,q \in \Z, \gcd(p,q) =1 \\
\Rightarrow 4p^2D=((2a)^2-4z)q^2$ \quad ($\star$)\\
$\Rightarrow q=1 $ or $2$.
\item \underline{Show}: $q=2 \Rightarrow D \equiv 1 \mod 4$\\
$(\star) \Rightarrow p^2D= (2a)^2 -4z \equiv (2a)^2 \mod 4$\\
$p$ is odd, hence $p^2 \equiv 1 \mod 4 \Rightarrow (2a)$ is odd (i.e. $a=\frac{2n-1}{2} \in \Q$)\\ $\Rightarrow (2a)^2 \equiv 1 \mod 4 \Rightarrow D \equiv 1 \mod 4$.
\item It follows from (2) if $D \equiv 1 \mod 4$:\\
$\alpha \in B \iff \alpha = a +b \sqrt{D}$ or $\alpha = \frac{1}{2}(a +b \sqrt{D})$ with $a,b \in \Z$. Hence we obtain:
\begin{align*}
B= \begin{cases}
\Z[\sqrt{D}] &, \text{ if } D \equiv 2,3 \mod 4\\
\Z[\frac{1}{2}(1+ \sqrt{D}] &, \text{ if } D \equiv 1 \quad \mod 4
\end{cases}
\end{align*}
For the second case observe that $\frac{a}{2} + \frac{b}{2} \sqrt{D} = \frac{a-b}{2} + \frac{b}{2} (1+ \sqrt{D}) \in \Z[\frac{1}{2}(1+ \sqrt{D}]$.\\
 This implies the claim.
\end{enumerate}
\end{Bew}

\begin{Prop}
Suppose $L/K$ separable and $A$ is a principal ideal domain. Let $M \not = 0$ be a finitely generated $B$-submodule of $L \Rightarrow M$ is a free $A$-module. In particular: $B$ is a free $A$-module of rank $n:=[L:K]$.
\end{Prop}

\begin{remin}
Suppose $A$ is a principal ideal domain and $M_0$ is a finitely generated free $A$-module.
\begin{enumerate}[i)]
\item Any submodule $M$ of $M_0$ is free.
\item $\rank(M_0) \geq \rank(M)$
\end{enumerate}
\end{remin}

\begin{Bew}[of Prop 2.12]
Let $\mu_1, \dots, \mu_r \in M \subseteq L$ be generators of $M$ as $B$-module and let $\alpha_1, \dots, \alpha_n$ be a basis of $L/K$ in $B$ and $d:=d(\alpha_1, \dots, \alpha_n) \in A$.\\
Recall: $L = \{ \frac{b}{a} \ | \ b \in B, a \in A \setminus \{0\}\}$.
\begin{enumerate}[(1)]
\item Prop 2.7 $\Rightarrow dB \subseteq A \alpha_1 + \dots + A \alpha_n$
\item $\exists a \in A: a \mu_1, \dots, a \mu_r \in B$
\end{enumerate}
Hence: $daM \subseteq dB \subseteq A \alpha_1 + \dots + A \alpha_n =: M_0 $\\
($M_0$ is a free $A$-module, since $\alpha_1, \dots, \alpha_n$ are basis of $L/K$).\\
Reminder 2.13 $\Rightarrow adM$ is a free $A$-module $\Rightarrow M$ is a free $A$-module.\\ Furthermore: $\rank(M)= \rank (adM) \stackrel{Rem. 2.13}{\leq } \rank (M_0) = n$.\\
Suppose that $M = B$. So far we got that $B$ is a free $A$-module and $\rank(B) \leq n$.\\
\underline{Show:} $\rank(B) \geq n$.\\
Let $\mu_1, \dots \mu_r$ be a basis of $B$ as $A$-module. By $L=\{\frac{b}{a} \ | \ b \in B, a \in A \setminus \{0\}\}$ we have that $\mu_1, \dots, \mu_r$ generate $L$ over $K$.
\end{Bew}

Hence: if $A$ is a principal ideal domain, then $B$ has always an integral basis.

\todo{Intermezzobild (?). $L,L'$ galois field extensions of $K$ of degree $n,m$ with $(n,m)=1$}
\begin{Prop}
Suppose we are in the following situation:
\begin{itemize}
\item $L/K$ and $L'/K$ are Galois extensions of degree $n$ and $m$ in some field $E$
\item $A$ a subring of $K$ such that $K=\Quot(A)$ and $B$ and $B'$ are the integral closures of $A$ in $L$ and $L'$.
\item $\{\omega_1, \dots, \omega_n\}$ and $\{\omega'_1, \dots, \omega'_m\}$ are integral basis for $B/A$ and $B'/A$.
\item $d:=d(\omega_1, \dots, \omega_n)$ and $d':=d(\omega'_1, \dots, \omega'_m) \in A$ with $d$ and $d'$ are coprime in $A$, i.e. $\exists x, x' \in A$ with $1=dx+d'x'$.
\item $K=L \cap L'$
\end{itemize}
Then we have: $\{\omega_i \omega'_j \ | \ i \in \{1, \dots, n \}, j \in \{1, \dots, m \}\}$ is an integral basis and its discriminant  is $d^m (d')^n$.
\end{Prop}

\begin{Bew}
Recall: $L \cap L' =K \Rightarrow [LL' :K]=nm$ and $\{\omega_i\omega'_j\}$ is a basis of the field extension $LL' / K$.\\
$\Gal(L/K)=\{\sigma_1, \dots, \sigma_n\}$ and $\Gal(L'/K) = \{\sigma'_1, \dots, \sigma'_m\}$\\
$\Rightarrow$ obtain unique lifts $\hat{\sigma}_i \in \Gal(LL'/L')$ and $\hat{\sigma}'_j \in \Gal(LL'/L)$ and $\Gal(LL'/K)=\{\hat{\sigma}_i \hat{\sigma}'_j \ | \ i \in \{1, \dots, n \}, j \in \{1, \dots, m\}\}$.\\
Consider: $\alpha \in \tilde{B}:= $ integral closure of $A$ in $LL'$.\\
Write $\alpha = \sum_{i,j} \alpha_{i,j} \omega_i \omega'_j = \sum_j \beta_j \omega'_j$ with $\alpha_{i,j} \in K$ and $\beta_j= \sum_i \alpha_{i,j} \omega_i \in L$.\\
$\Rightarrow \hat{\sigma}'_i(\alpha) = \sum_j \beta_j \hat{\sigma}'_i(\omega'_j)$, since $\hat{\sigma}'_i \in \Gal(LL'/L)$.\\
$\Rightarrow$ We have a linear system:
\begin{align*}
a=Tb \text{ with } a = \begin{pmatrix}
\hat{\sigma}'_1(\alpha)\\
\vdots\\
\hat{\sigma}'_m(\alpha)
\end{pmatrix}
\in \tilde{B}^m \ , \ b = \begin{pmatrix}
\beta_1\\
\vdots\\
\beta_m
\end{pmatrix} \in L^m
\ ,\ T=(\hat{\sigma}'_i(\omega'_j))_{(i,j)} \in \tilde{B}^{m \times m}
\end{align*}
Observe: $\det(T)^2 = d'$
\begin{align*}
\&Rightarrow \det(T) b = T^{\#} T b = T^{\#}a \in \tilde{B}^m &\Rightarrow d'b \in \tilde{B}^m\\
&\Rightarrow \forall j: d' \beta_j = \sum_i d'\alpha_{i,j} \omega_i \in \tilde{B} \cap L =B\\
&\Rightarrow d' \alpha_{i,j} \in A, \text{ since } \{\omega_1, \dots, \omega_n\} \text{ is an integral basis}.\\
&\Rightarrow d\alpha_{i,j} \in A \text{ in the same way}\\
&\Rightarrow \alpha_{i,j} = (x'd'+xd) \alpha_{i,j}=x'd'\alpha_{i,j}+xd\alpha_{i,j} \in A.
\end{align*}
Hence: $\{\omega_i \omega'_j \ | \ (i,j) \in \{(1,1), \dots, (n,m)\}\}$ is an integral basis of $\tilde{B}/A$.\\
For calculating the discrimant consider the matrix $M= ( \hat{\sigma}_k \circ \hat{\sigma}'_l (\omega_i \omega'_j))_{(k,l),(i,j)} = (\hat{\sigma}_k(\omega_i) \hat{\sigma}'_l(\omega'_j))$.\\
Consider $Q= (\hat{\sigma}_k(\omega_i))$
\[
\Rightarrow M= \begin{pmatrix}
Q &0 &\dots &0\\
0 &\ddots & &\vdots\\
\vdots &&\ddots &0\\
0 &\hdots &0 &Q
\end{pmatrix}
\cdot
\begin{pmatrix}
I\cdot \hat{\sigma}'_1(\omega'_1) &\cdots &I\cdot &\hat{\sigma}'_1(\omega'_1)\\
\vdots & &\vdots\\
\vdots &  &\vdots\\
I\cdot \hat{\sigma}'_1(\omega'_m) &\cdots &I\cdot &\hat{\sigma}'_m(\omega'_m)\\
\end{pmatrix}
\]
\underline{Observe:}
\begin{enumerate}[(1)]
\item $\det(Q)^2=d(\omega_1, \omega_n)=d$
\item The second matrix can be transformed by switching rows and columns to $\begin{pmatrix}
Q' &0 &\dots &0\\
0 &\ddots & &\vdots\\
\vdots &&\ddots &0\\
0 &\hdots &0 &Q'
\end{pmatrix}$ with $Q'=(\sigma'_l(\omega'_j))$ and $\det(Q')=d'$
\end{enumerate}
$\Rightarrow \det(M)^2=\det(Q)^{2m} \cdot \det(Q')^{2n} = d^m d'^n$.
\end{Bew}
