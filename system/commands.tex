% ---------- Definitionen, Sätze, ... ----------
\newtheoremstyle{defistyle}% hnamei
{}% hSpace abovei
{}% hSpace belowi
{}% hBody fonti
{}% hIndent amounti1
{\bfseries}% hTheorem head fonti
{}% hPunctuation after theorem headi
{\newline}% hSpace after theorem headi2
{}% hTheorem head spec (can be left empty, meaning `normal')i


      %\theoremstyle{plain}
      \theoremstyle{defistyle}
      \newtheorem{theorem}{Theorem}
      \newtheorem*{theorem*}{Theorem}
      \newtheorem{lemma}[theorem]{Lemma}
      \newtheorem*{lemma*}{Lemma}
      \newtheorem{cor}[theorem]{Corollary}
      \newtheorem*{cor*}{Corollary}
      \newtheorem{prop}[theorem]{Proposition}
      \newtheorem*{prop*}{Proposition}
      \newtheorem{conjecture}[theorem]{Conjecture}  
      \newtheorem*{conjecture*}{Conjecture}   
      \newtheorem{claim}[theorem]{Claim}   
      
            
\newtheorem{defi}[theorem]{Definition}
\newtheorem{Bem}[theorem]{Remark}
\newtheorem*{Bem*}{Remark}
\newtheorem{Not}[theorem]{Notation}
\newtheorem{Bemdef}[theorem]{Remark and Definition}
\newtheorem{Lemdef}[theorem]{Lemma and Definition}
\newtheorem{Satz}[theorem]{Theorem}
\newtheorem{Lem}[theorem]{Lemma}
\newtheorem{Bsp}[theorem]{Example}
\newtheorem{Kor}[theorem]{Corollary}
\newtheorem{Prop}[theorem]{Proposition}
\newtheorem{idea}[theorem]{Idea}
\newtheorem{prob}[theorem]{Problem}
\newtheorem{remin}[theorem]{Reminder}
\newtheorem{Beobachtung}[theorem]{Beobachtung}
\newtheoremstyle{bewstyle}
{}% hSpace abovei
{}% hSpace belowi
{}% hBody fonti
{}% hIndent amounti1
{\bfseries}% hTheorem head fonti
{}% hPunctuation after theorem headi
{}% hSpace after theorem headi2
{}% hTheorem head spec (can be left empty, meaning `normal')i
\newtheorem*{Bew}{Proof:}
      
      \theoremstyle{definition}
      
      \newtheorem{definition}[theorem]{Definition}
      \newtheorem*{definition*}{Definition}
      
      \theoremstyle{remark}
      
      \newtheorem{remark}[theorem]{Remark}
      \newtheorem*{remark*}{Remark}     
      \newtheorem{example}[theorem]{Example}
      \newtheorem*{example*}{Example}
      \newtheorem{reminder}[theorem]{Reminder}
      
      \numberwithin{theorem}{section}
      
      



% ========== Abkürzungen ==========

% ---------- Mengen, Buchstaben, ... ----------

\newcommand{\N}{\mathbb{N}}
\newcommand{\Z}{\mathbb{Z}}
\newcommand{\Q}{\mathbb{Q}}
\newcommand{\R}{\mathbb{R}}
\newcommand{\C}{\mathbb{C}}
\let\k\relax
\newcommand{\k}{\mathbb{K}}
\newcommand{\Aff}{\mathbb{A}}

\newcommand{\Pot}{\mathcal{P}}

\let\P\relax
\newcommand{\P}{\mathbb{P}}

\newcommand{\E}{\mathcal{E}}
\newcommand{\F}{\mathcal{F}}
\newcommand{\G}{\mathcal{G}}

\newcommand{\A}{\mathcal{A}}
\newcommand{\B}{\mathcal{B}}

\let\O\relax
\newcommand{\O}{\mathcal{O}}

\newcommand{\Le}{L^1}
\newcommand{\SLe}{\mathscr{L}^1}
\newcommand{\Cpi}{\mathcal{C}_{2\pi}}

\newcommand{\p}{\mathfrak{p}}

\newcommand{\X}{\mathscr{X}}
\newcommand{\U}{\mathcal{U}}
\newcommand{\LC}{\mathcal{L}}

\newcommand{\one}{\mathbbm{1}}

\DeclareMathOperator{\BV}{BV}

\newcommand{\K}{\mathcal{K}}
\newcommand{\D}{\mathbb{D}}


% ---------- Funktionen ----------

\newcommand{\abs}[1]{\left|{#1}\right|}
\newcommand{\norm}[1]{\left\|{#1}\right\|}
\newcommand{\Lenorm}[1]{\left\|{#1}\right\|_{\Le}}
\newcommand{\SLenorm}[1]{\left\|{#1}\right\|_{\SLe}}
\newcommand{\Bnorm}[1]{\left\|{#1}\right\|_{B}}
\newcommand{\supnorm}[1]{\left\|{#1}\right\|_{\infty}}
\newcommand{\lipnorm}[1]{\left\|{#1}\right\|_{\text{Lip}}}
\newcommand{\BVnorm}[1]{\left\|{#1}\right\|_{\BV}}

\newcommand{\bigO}[1]{\mathcal{O}\left(#1\right)}
\newcommand{\littleo}[1]{o\left(#1\right)}


\newcommand{\ceil}[1]{\left\lceil {#1} \right\rceil}
\newcommand{\floor}[1]{\left\lfloor {#1} \right\rfloor}


\newcommand{\ex}[1]{\mathbb{E}\left[ {#1} \right]}
%\newcommand{\prob}[1]{\mathbb{P}\left[ {#1} \right]}


% ---------- MathOperators ----------

\let\Re\relax
\let\Im\relax
\DeclareMathOperator{\Re}{Re}
\DeclareMathOperator{\Im}{Im}
\DeclareMathOperator{\Int}{Int}
\DeclareMathOperator{\Gl}{Gl}
\DeclareMathOperator{\Var}{Var}
\DeclareMathOperator{\Cov}{Cov}
\DeclareMathOperator{\Corr}{Corr}
\DeclareMathOperator{\Mat}{Mat}
\DeclareMathOperator{\GL}{GL}
\DeclareMathOperator{\SL}{SL}
\DeclareMathOperator{\Sp}{Sp}
\DeclareMathOperator{\ind}{ind}
\DeclareMathOperator{\res}{res}
\DeclareMathOperator{\supp}{supp}
\DeclareMathOperator{\Syz}{Syz}
\DeclareMathOperator{\sgn}{sgn}
\DeclareMathOperator{\diam}{diam}
\DeclareMathOperator{\ord}{ord}
\DeclareMathOperator{\spec}{spec}
\DeclareMathOperator{\Bild}{Bild}
\DeclareMathOperator{\dist}{dist}
\DeclareMathOperator{\rg}{rg}
\DeclareMathOperator{\tr}{tr}
\DeclareMathOperator{\diag}{diag}
\DeclareMathOperator{\id}{id}
\DeclareMathOperator{\spanset}{span}
\DeclareMathOperator{\LH}{LH}
\DeclareMathOperator{\proj}{proj}
\DeclareMathOperator{\Log}{Log}
\DeclareMathOperator{\ggT}{ggT}
\DeclareMathOperator{\Quot}{Quot}
\DeclareMathOperator{\Hom}{Hom}
\DeclareMathOperator{\Gal}{Gal}
\DeclareMathOperator{\Norm}{\mathcal{N}}
\newcommand{\TrL}{\operatorname{Tr_{L/K}}}
\newcommand{\Tr}{\operatorname{Tr}}
\DeclareMathOperator{\NormL}{\mathcal{N}_{L/K}}
\DeclareMathOperator{\rank}{rank}



% ---------- Sonstiges ----------

\newcommand{\piint}{\frac{1}{2\pi}\int\limits_{0}^{2\pi}}

\newcommand{\td}{\,\mathrm{d}}
\newcommand{\bs}{\backslash}